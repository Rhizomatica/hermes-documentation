\documentclass[11pt,a4paper]{article}
\usepackage[brazil]{babel} % carrega portugues brasileiro
\usepackage[utf8]{inputenc}
\usepackage[T1]{fontenc}
\usepackage[top=2cm, bottom=2cm, left=2cm, right=2cm]{geometry} %margens menores!
\usepackage{graphicx} % incluir figuras .eps
\usepackage{tabularx}
\usepackage{color} % colorir texto
\usepackage{indentfirst}
\usepackage{textcomp}
\usepackage[colorlinks=true]{hyperref}
\usepackage{amssymb,amsmath}
\usepackage{float}
% \usepackage{siunitx}
% \usepackage[ampersand]{easylist}

\title{Descrição De Software do HERMES - High-frequency Emergency and Rural Multimedia Exchange}

\author{
       \large
        \textsc{Rafael Diniz}
        \mbox{}\\ %
        rafael@rhizomatica.org\\
        \mbox{Rhizomatica} \\ %
%        \normalsize
%        \texttt{Brasília - Brasil}\\
}
\date{\today}


\begin{document}

\maketitle

\begin{abstract}
Esta é a descrição da pilha de software do sistema de telecomunicação digital em HF HERMES. O HERMES
utiliza um modem para HF de alta performance e o sistema UUCP (Unix to Unix
Communication Protocol). Este documento também abrange a descrição de
software da interface gráfica para usuário baseada em tecnologia Web, e o sistema de transporte de emails
que o HERMES provê.

\end{abstract}

\newpage

\tableofcontents

\section{Sistema de referência}

O Rhizomatica desenvolveu o sistema HERMES, que contém, em suma, um transceptor HF conectado a um computador.
Outras opções para se construir um sistema compatível com o HERMES seria através do uso de um transceptor de prateleira, como qualquer rádio com porta USB como o ICOM IC-7100 ou o ICOM IC-7300, ou opcionalmente o ICOM IC-78, ICOM IC-718 ou Vertex
VX-1700 com uma interface externa (eg. Signalink USB), conectado a um computador rodando Linux.

Para o sistema de alimentação elétrica com baterias e painel solal:
\begin{itemize}
\item Bateria Estacionária 150Ah. Referência: \url{https://www.neosolar.com.br/loja/bateria-estacionaria-moura-clean-12mf150-150ah.html}
\item Um ou dois paineis solares. O ideal seria que juntos totalizem 250W ou mais de potência. 150W é o mínimo aceitável. Referência: \url{https://www.neosolar.com.br/loja/painel-solar-fotovoltaico-155wp-upsolar-up-m155p.html}
\item Controlador de carga de no mínimo 30A. 40A recomendado. Referência: \url{https://www.neosolar.com.br/loja/controlador-carga-pwm-30a-12-24v-epever-landstar-ls3024eu.html}
\end{itemize}

Para alimentação com energia da rede 110 ou 220V AC, é necessária uma fonte de 13.8V ou 12V.
\begin{itemize}
\item Fonte linear de trafo 30A 13.8V. Referência: \url{https://www.radiohaus.com.br/produto/8498/fonte-maxtron-30a-estabilizada} OU
\item Fonte chaveada 50A 12V de baixo custo. Referência:
  \url{https://produto.mercadolivre.com.br/MLB-1240884918-fonte-chaveada-12v-50a-600w-s600-12-cftv-led-bivolt-_JM?quantity=1#position=1&type=item&tracking_id=9596443b-38ac-416f-ab5c-d1adf108e56d}
\end{itemize}

Para o rádio transceptor transmitir e receber ondas de rádio, uma antena ajustada para a frequência de transmissão é necessária.

Exemplo de antena a ser utilizada:
\begin{itemize}
\item Antena dipolo de fio com BALUN feita sob medida para a frequência escolhida, instalada como V invertida. Referências: Electril:     \url{http://www.electril.com/electril/dipolo2.htm} ML: \url{https://produto.mercadolivre.com.br/MLB-1435429491-tw713-antena-dipolo-para-40-e-20-metros-com-balun-11-_JM?quantity=1#position=1&type=item&tracking_id=771c82d5-2426-46d9-8cbb-d34b6e1ec2d1}
\end{itemize}

Para o caso de rádio-tranceptores HF para radioamadorismo, todos eles precisam ser "desbloqueados" (exceção do IC-78 - mas sempre confirmar com o vendedor) para operar em faixas fora da banda de rádio amador. Recomenda-se o uso do transceptor de rádio do Rhizomatica para o uso do sistema HERMES.

Segue uma lista rádio-transceptores de prateleira que são mais fáceis para se adicionar comunicação digital pois já possuem porta USB e placa de som embutida (entre parênteses as bandas suportadas - precisamos somente HF).

\begin{itemize}
\item ICOM IC-7100 (HF / VHF / UHF)
\item ICOM IC-7300 (HF)
\item Yaesu FT-991A (HF)
\end{itemize}

Segue uma lista com rádios mais baratos, mas que necessitam de uma interface especial para comunicação digital (ex: Tigertronics Signalink USB):
\begin{itemize}
\item ICOM IC-78 (HF)
\item ICOM IC-718 (HF)
\item Vertex VX-1700 (HF)
\item Alinco DX-SR8T (HF)
\item Yaesu FT-857D (UHF + VHF + HF)
\item Yaesu FT-891 (HF)
\end{itemize}

Exemplos de interfaces externas para conexão do rádio a um computador, para rádios sem conexão USB (Universal Serial Bus) e conversores AD/DA internos (conexão com PC entre parênteses):
\begin{itemize}
\item Tigertronics Signalink USB (USB)
\item DigiMaster MiniProSC (USB)
\item West Mountain Radio RIGblaster (USB e Bluetooth)
\end{itemize}

Cabo coaxial com 50ohm de impedância, podendo ser o RG-58, para curta distância ou RG-213, para mais longa distância.

Os cabos de energia deve ter bitola grande, suficiente para passar picos de mais de 20A em 12V.

Equipamentos essenciais para teste:
\begin{itemize}
\item Wattimetro com medição de potência direta e refletida para HF. Referência: \url{https://radiohaus.com.br/produto/4064/mfj-822-wattimetro-hf-vhf-18-200-mhz-300-w-movel}
\item Multímetro. Referência: \url{https://produto.mercadolivre.com.br/MLB-681832769-multimetro-digital-portatil-et-1002-minipa-_JM?quantity=1#position=1&type=item&tracking_id=5bd7e5cd-d473-442f-9203-57b77e67a56b}
\item Carga fantasta (dummy load). Referências: \url{https://www.radiohaus.com.br/produto/2969/mfj-262b-carga-fantasma-200w-0-a-1ghz} ou \url{https://produto.mercadolivre.com.br/MLB-912917571-carga-rf-100w-50r-carga-fantasma-preciso-50-ohms-2ghz-_JM#position=15&type=item&tracking_id=cfa216de-0679-46b0-9f31-480e35dd0981}
\end{itemize}


\section{Configuração}

O ARDOP é um modem SDR feito por rádio amadores que utiliza modernas
técnicas de modulação (OFDM) e UUCP é uma sistema para comunicação
assíncrona da década de 70 para sistemas UNIX, muito utilizado até
hoje em nichos, como comunicação em HF.

Debian Buster (10) arm64 (multilib com armhf no caso se querer usar os
binários ``piardopc'' do John Wiseman) é o sistema Linux de referência,
rodando numa Raspberry Pi 3 com Wifi configurado no modo AP (Access Point),
para provimento de interface do sistema via Web.

Para integrar o sistema UUCP ao modem ARDOP, o projeto RHIZO-UUARDOPD
foi desenvolvido para prover ferramentas que integram o UUCP ao ARDOP.

\subsection{Radio e/ou interface}

No caso de uso da interface Signalink, o ajuste de ``delay'' deve ser
zerado.

No caso dos ICOM IC-7100, deixar o filtro maior que 2.8kHz para o mode
SSB/Data em uso com o sistema digital.

\subsection{Alsa}

Add to ``/etc/asound.conf'':
\begin{verbatim}
pcm.ARDOP {type rate slave {pcm "hw:1,0" rate 48000}}
\end{verbatim}

\subsection{ARDOP}

Link para download: \url{http://www.cantab.net/users/john.wiseman/Downloads/Test/TeensyProjects.zip}

O binário utilizado do ardop deve ficar em /usr/bin/ardop, que pode ser um
link simbólico para /usr/bin/{ardop1ofdm, ardop2, ardopofdm}, por exemplo.

Configuração do ardop para uso com o ICOM IC-7100 (via porta USB, com PTT
via porta serial):
\begin{verbatim}
[Unit]
Description=ARDOP daemon

[Service]
Type=simple
ExecStart=/usr/bin/ardop 8515 -c /dev/ttyUSB0 ARDOP ARDOP -k FEFE88E01C0001FD -u FEFE88E01C0000FD
ExecStop=/usr/bin/killall -s QUIT ardop
IgnoreSIGPIPE=no
#StandardOutput=null
#StandardError=null
StandardOutput=syslog
StandardError=syslog

[Install]
WantedBy=multi-user.target
\end{verbatim}

Configuração para uso com a interface Signalink (VOX):
\begin{verbatim}
[Unit]
Description=ARDOP daemon

[Service]
Type=simple
ExecStart=/usr/bin/ardop 8515 ARDOP ARDOP
ExecStop=/usr/bin/killall -s QUIT ardop
IgnoreSIGPIPE=no
#StandardOutput=null
#StandardError=null
StandardOutput=syslog
StandardError=syslog

[Install]
WantedBy=multi-user.target
\end{verbatim}


Iniciar / Parar serviço:
\begin{verbatim}
systemctl start ardop.service
systemctl stop ardop.service
\end{verbatim}


Ver log:
\begin{verbatim}
journalctl -f -u ardop
\end{verbatim}

\subsection{UUCP}

Usar versão do pacote do debian 1.07-27 ou superior, por exemplo, a versão
do Debian Bullseye: \url{https://packages.debian.org/bullseye/uucp}.

ps: baixar o .deb e instalar na mão.

\subsection{rhizo-uuardopd}

%TODO: It seems everything uucico wrote in the end of a reception, don't get
%to the other side - connection closes and buffer get clean fast!

Dois binários, uuport (para ser usado pelo uucp) e uuardopd que é o software
que conecta com o modem Ardop através do uucico ou uuport.

\url{http://github.com/DigitalHERMES/rhizo-uuardop}


Iniciar / Parar serviço:
\begin{verbatim}
systemctl start uuardopd.service
systemctl stop uuardopd.service
\end{verbatim}


Ver log:
\begin{verbatim}
journalctl -f -u uuardopd
\end{verbatim}

%\subsection{DHCP}

%\subsection{DNS}

%\subsection{Apache + PHP}

\section{Interface via UUCP em linha de comando}

O diretório para arquivos recebidos ``/var/www/html/arquivos''.

Adicionando a fila de envio (``-r'' não inicia a transmissão de forma
automática):
\begin{verbatim}
uucp -C -r -d source.xxx AM4AAB\!/var/www/html/arquivos/${nodename}/
\end{verbatim}

Para disparar o envio de todos os jobs para o destinatário
AM4AAA:
\begin{verbatim}
uucico -S AM4AAA
\end{verbatim}

Para lista os jobs:
\begin{verbatim}
uustat -a
\end{verbatim}

Para matar jobs:
\begin{verbatim}
uustat -k job
uustat -K
\end{verbatim}

Ver log:
\begin{verbatim}
uulog
\end{verbatim}

\section{Interface com o usuário}

A interface do sistema é baseada em Web, utilizando HTML+PHP e
alguns shell scripts.

\section{Pré-requisitos}

\begin{itemize}
\item ImageMagic: para descomprimir imagens
\item mozjpeg: para comprimir em jpg: https://github.com/mozilla/mozjpeg
% \item opusenc: para comprimir áudio
\item gpg: para criptografia
\item hostapd: para o roteador wifi (com senha ou sem senha)
\end{itemize}

\subsection{Interface Web}

Acessível digitando qualquer endereço no navegador, ou 192.168.1.1.

\subsection{WebPhone}

WIP!

\url{https://gitlab.tic-ac.org/keith/webphone/wikis/hermes}

\end{document}
