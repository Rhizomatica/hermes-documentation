\documentclass[11pt,a4paper]{article}
% \usepackage[brazil]{babel} % carrega portugues brasileiro
\usepackage[utf8]{inputenc}
\usepackage[T1]{fontenc}
\usepackage[top=2cm, bottom=2cm, left=2cm, right=2cm]{geometry} %margens menores!
\usepackage{graphicx} % incluir figuras .eps
\usepackage{tabularx}
\usepackage{color} % colorir texto
\usepackage{indentfirst}
\usepackage{textcomp}
\usepackage[colorlinks=true]{hyperref}
\usepackage{amssymb,amsmath}
\usepackage{float}
% \usepackage{siunitx}
% \usepackage[ampersand]{easylist}

\title{HERMES Hardware Description}

\author{
       \large
        \textsc{Rafael Diniz}
        \mbox{}\\ %
        rafael@rhizomatica.org\\
        \mbox{Rhizomatica} \\ %
%        \normalsize
%        \texttt{Brasília - Brasil}\\
}
\date{\today}


\begin{document}

\maketitle

\begin{abstract}
This document describes the hardware list and configuration needed to set up
a HERMES node, which consists of a HF station capable
of sending and receiving digital data. This document is based on field
experiences acquired through HF systems deployed in the south of Mexico
and in the Amazon region in South America. The software stack for digital
telecommunications system is described in another document.
\end{abstract}

\newpage

\tableofcontents

\section{HERMES Hardware}

The HERMES setup is composed by different elements. The HF
transceiver is responsible for receiving and transmitting signals
over-the-air, allowing the exchange of digital data. The transceiver
is necessarily connected to an antenna and a computer. Also a power source
is needed. The list of main components for HERMES are the following:

\begin{itemize}
\item HF Radio Transceiver
\item Antenna
\item Mini-Computer
\item Power source
\end{itemize}

Each part has its own dedicated subsection, and also a section with
important tools to have in order to properly adjust the HF station is
present.

\subsection{HF Radio Transceiver}

Many options for HF transceiver are available, but in order to allow easy
operation and better integration of the overall HERMES system, Rhizomatica
developed its own HF transceiver, based on the $\mu$Bitx v6 board (See:
\url{https://www.hfsignals.com/index.php/ubitx/}).

If not using Rhizomatica's transceiver, we consider best options for HF
radio transceiver the ones which have a USB (Universal Serial Bus) port and internal
analog/digital converters, which provide seamless and error-free
connection between the HF transceiver and a computer. Taking in
consideration the cost of the transceivers, usually the most affordable ones
are made for the amateur (ham) radio market. Ham radio transceivers are
usually blocked to transmit only inside the ham radio allocated bands, and need
to be modified to transmit in all the HF bands (See example here:
\url{https://radioaficion.com/cms/ic-7100-marscap-modification/}).

In order to allow HERMES modem to work, a pass-band of at least 2.7kHz is
recommended in SSB mode (the mode typically used for data and voice
transmissions in HF).

Among this category we list as examples (supported frequency bands between parenthesis):

\begin{itemize}
\item ICOM IC-7100 (HF / VHF / UHF)
\item ICOM IC-7300 (HF)
\item Yaesu FT-991A (HF)
\end{itemize}

Other HF transceivers can also be used for digital telecommunications, but
need an external interface to connect the radio to a computer. Examples
of transceivers in the category are:

\begin{itemize}
\item ICOM IC-7000 (HF / VHF / UHF)
\item ICOM IC-78 (HF)
\item ICOM IC-718 (HF)
\item Vertex VX-1700 (HF)
\item Alinco DX-SR8T (HF)
\item Yaesu FT-857D (HF / VHF / UHF)
\item Yaesu FT-891 (HF)
\end{itemize}

If an external interface is needed to connect a transceiver to computer, be
aware that an adjust procedure to identify the correct audio levels to
correctly drive the transceiver and to optimally receive the audio needs to
be done. With some transceivers like the ICOM IC-78 and IC-718 there is a
need to set the MIC gain level to 0 when transmitting using its ACC
connector. Different transceivers have different quirks and issues, so be
aware to always carry tests before taking any equipment to the field. This
is one of the reasons Rhizomatica took to path to develop its own
transceiver.

Examples of external interfaces which support all transceiver
models (PC connection between parenthesis):
\begin{itemize}
\item Tigertronics Signalink USB (USB)
\item DigiMaster MiniProSC (USB)
\item West Mountain Radio RIGblaster (USB and Bluetooth)
\end{itemize}

\subsection{Antenna}

The antenna is a very special part of the system. If not well tuned or well
installed, nothing works. We recommend the use of the appropriate antenna
for the desired coverage. For Near Vertical Incidence Skywave (NVIS), which
provides up to 600km radius of coverage, we recommend a simple quarter wavelength
antenna with a balun for impedance matching. If a portable antenna is
needed, the Buddipole antenna is a good option, but with worse performance
then a simple 1/4 wavelength dipole (See
\url{https://www.buddipole.com/}). Frequency bands below 7MHz are
recommended for NVIS operation.

For wider coverage and more-or-less omnidirectional radiation pattern, the
installation of a dipole as inverted V is desired.

\subsection{Mini-computer}

The mini-computer will host the HF modem application, the network
stack and services. In the minimal setup, the computer will run the HF
modem application and a Web server which provides access to the HF
telecommunication system services over a WiFi network. A Raspberry Pi 4 has
enough processing power and memory for this use case. For a proper HERMES
installation, with full email services, automatic media compression and WiFi
access point, the following configuration is recommended:

\begin{itemize}
\item Intel-based computer (4th generation or better);
\item 4GB of RAM;
\item 120 GB of storage (PCIe NVMe highly recommended) is needed;
\item WiFi adapter with support to access point mode (M.2 type recommended);
\item 12V to ATX power supply (for off-the-grid use cases).
\end{itemize}

Optionally, instead of WiFi for user access to services, HERMES can host
a mobile network access point, for example, using GSM or LTE technologies.

\subsection{Power source}

A common HF transceiver uses a 12V voltage input and at least 1A current in
receiving mode, and peaks up to 25A when transmitting. So it is important
to have a charge controller which can provide at least 30A, in order to
allow some headroom for fluctuations of consumption.

As a example scenario, take a transceiver using 20A when transmitting full
power, 1A while listening. Lets assume we want to transmit for 5 minutes
every 24 hours and listen all the time when we're not transmitting.

The daily power budget of the system (@12V) is:\\
HF rx: $1A * 24h = 24Ah$ \\
HF tx: $20A * 5/60h = 100/60 = 1.7Ah$\\
mini-pc: $0.5A (~ 1.2A@5V) * 24h = 12Ah$\\

So total daily battery consumption would be $24+1.7+12 = 37.7Ah$

Let's say we want to discharge our batteries 15\% on a normal day, so
this load would represent a battery capacity of 100/15 * 37.7 = 251.3Ah.

We'd probably also want some spare capacity, say, an extra day of using the 
system when there's no sun. (solar panels are really bad at producing output
on cloudy days: expect 8\%-20\% of normal output). So we'll either have to
add more batteries or stop listening all day (nearly all our power
consumption is for listening, not talking). for the solar panels recharging
the battery, it is a good practice to size things at somewhere around 2
times the normal load every day, so that the system recharges in about 1 day
after 1 day of operating without sun, plus add around 15\%-20\% for
inefficiencies.

The solar maps for the Amazon region show a yearly average insolation
of about $4.0 kWh/m^2/d$. Based on this,\\
$4h * Cpanel(W) = 37.7Ah*12.5V*2.2$\\
$Cpanel(W) = 2.2*(37.7Ah*12.5V)/(4h)$\\
$= 259.2W$

So there is a need for around 250W to 300W of solar panels.

Another good practice would be to double the battery capacity for this
scenario to make the system workable for more than 1 sunless day without
discharging the batteries too deeply.

Recommended power related parts:
\begin{itemize}
\item 250Ah stationary battery
\item One 250W or 300W solar panel, or two 250W panels
\item Charge controller which can output at least 30A of current. 40A
  recommended.
\end{itemize}

In the case of using the HF transceiver connected to power grid, a AC/DC
power supply is needed, with 12V or 13.8V output. A couple of options for
the AC/DC power supply:


\begin{itemize}
\item At least 30A transformer-based power supply at 12V or 13.8V.
\item Low cost switched power supply at 12V or 13.8V marked as 50A or more.
\end{itemize}

\subsection{Essential tools}

Essential tools to have for basic radio and antenna testing:
\begin{itemize}
\item Wattmeter with reflected and forward power readings for HF, which
  supports at least 100W
\item Multimeter
\item 50ohm dummy load for at least 100W
\end{itemize}


\end{document}
