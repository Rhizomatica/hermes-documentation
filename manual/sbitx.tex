\documentclass[11pt,a4paper]{article}
% \usepackage[brazil]{babel} % carrega portugues brasileiro
\usepackage[utf8]{inputenc}
\usepackage[T1]{fontenc}
\usepackage[top=2cm, bottom=2cm, left=2cm, right=2cm]{geometry} %margens menores!
\usepackage{graphicx} % incluir figuras .eps
\usepackage{tabularx}
\usepackage{color} % colorir texto
\usepackage{indentfirst}
\usepackage{textcomp}
\usepackage[colorlinks=true]{hyperref}
\usepackage{amssymb,amsmath}
\usepackage{float}
% \usepackage{siunitx}
% \usepackage[ampersand]{easylist}

\title{HERMES Hardware Description}

\author{
       \large
        \textsc{Rafael Diniz}
        \mbox{}\\ %
        rafael@rhizomatica.org\\
        \mbox{Rhizomatica} \\ %
%        \normalsize
%        \texttt{Brasília - Brasil}\\
}
\date{\today}


\begin{document}

\maketitle

\begin{abstract}
  This document describes the software setup of the sBitx v2 HF transceiver.
  The sBitx v2 radio is an ahead-of-its-time wide-band HF transceiver, fully
  controlled by software. The radio is composed by the analog radio-frequency circuitry,
  Raspberry Pi 4 (RPi 4), ATTiny85 microcontroller (for lambda bridge FWD and REF readings), battery-backed
  Real-Time-Clock, Wolfson wm8731 audio codec, 7 inches high-def touch screen and two knobs with `` button'',
\end{abstract}

\newpage

\tableofcontents

\section{Introduction}

This document address all the software details needed for the sBitx v2 hardware bringup.



\section{sBitx v2 Hardware}


\subsection{Power consumption}


\section{Hardware interfaces to the computer (RPi 4)}


\subsection {Control I/O}

\subsubsection{I2C RTC}

\subsubsection{I2C Si5351}

\subsubsection{I2C ATtiny85 (FWD and REF)}

\subsubsection{Digital pins setup for knobs control}

\subsection{Signal I/O}

\subsubsection{I2S Wolfson wm8371}

\section{Linux setup for sBitx operation}

\subsection{Pin attribution setup for the RPi 4}

\subsection{Audio - ALSA subsystem}

\subsection{Ashhar's sbitx reference implementation}

\subsubsection{power setup}

\subsection{bridge setup}

\subsection{I2C bit-banging}

\subsection{RTC setup}

\end{document}
